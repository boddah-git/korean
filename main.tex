\documentclass[12pt, a4paper, openany]{book}

% Must be compiled with XeLaTeX

% Font & CJK
\usepackage{fontspec}
\newfontfamily\ipafont{Charis SIL}
\usepackage{xeCJK}

\setCJKsansfont{DotumChe}
\setCJKmainfont[
  Path = ./fonts/,
  UprightFont = NotoSansKR-Regular.ttf,
  BoldFont = NotoSerifKR-Bold.ttf,
  SansFont = NotoSerifKR-Regular.ttf
]{NotoSansKR}

% Math
\usepackage{amsmath,amssymb}

% Layout
\usepackage{graphicx}
\usepackage{geometry}
\usepackage{titlesec}
\usepackage{multicol}
\geometry{margin=1in}

% Hyperlinks and bookmarks
\usepackage{hyperref}
\usepackage{bookmark}

% Color boxes
\usepackage[most]{tcolorbox}

\usepackage{tabularx, cellspace}
\usepackage{menukeys}
\usepackage{indentfirst}
\usepackage{glossaries}
\usepackage{tikz}
\usepackage{array}
\def\checkmark{\tikz\fill[scale=0.4](0,.35) -- (.25,0) -- (1,.7) -- (.25,.15) -- cycle;}

\titleformat{\chapter}[hang]{\normalfont\huge\bfseries}{\thechapter.}{1em}{}

% Custom Korean vocab box
\tcbset{
  box/.style={
    enhanced,
    attach boxed title to top center={yshift=-3mm,yshifttext=-1mm},
    title=#1
  }
}

\newcolumntype{C}{|X|}

\newcommand{\ipa}[1]{{\ipafont /#1/}}

\newcommand{\spacedstack}[1]{\vspace{0.3ex}\shortstack{#1}\vspace{0.3ex}}

\newcommand\rom[3][]{
  \ifx\relax#1\relax
    $\overset{\text{\color{red}#3}}{\text{#2}}$
  \else
    $\underset{\textbf{#1}}{\overset{\text{\color{red}#3}}{\text{#2}}}$
  \fi
}

\newcommand{\cross}{$\times$}

\setcounter{chapter}{-1}

\begin{document}

\begin{titlepage}
    \centering
    \vspace*{2cm}
    {\Huge\bfseries Korean\par}
    \vspace{1.5cm}
    {\Large Lautaro Bonseñor\par}
    \vfill
    {\large \today\par}
\end{titlepage}

\tableofcontents
\chapter{Basics}
\section{Korean Alphabet (한글)}
\subsection{Introduction}
Hangul (한글) is the writing system used for the Korean language. Similarly to Latin, it is a \textbf{phonetic alphabet} unlike Chinese or Japanese, which are logographic scripts. 

\subsection{Basic Structure}
Each character represents a \textbf{syllable} or \textbf{syllabic block}, and each syllable is composed of individual letters called \textbf{jamo (자므)}, which represent consonants or \textbf{jaeum (자음)} and vowels or \textbf{moeum (모음)}, read from left to right and top to bottom.

There are a total of 14 basic consonants and 10 monophthong vowels, following 5 double consonants and 11 additional vowels formed by combining two monophthongs in the same syllable (diphthongs).

Each syllable typically consists of:

\begin{itemize}
    \item An initial consonant \textbf{(Choseong 초성)}
    \item A medial vowel \textbf{(Jungseong 중성)}
    \item An optional final consonant \textbf{(Jongseong 종성)}
\end{itemize}

Something to note is that the vowels ㅏ,ㅓ,ㅑ,ㅕ,ㅐ,ㅔ,ㅖ and ㅒ are written to the \textbf{right} of the initial consonant, while the rest are written to the \textbf{bottom}.

Throughout this text, in order to facilitate the learning process, all Hangul writing will be written with the following format:

\begin{center}
    \rom{안녕하세요}{annyeonghaseyo}
\end{center}

Or, in case there is the need for translation:

\begin{center}
    \rom[hello]{안녕하세요}{annyeonghaseyo}
\end{center}

\newpage
Here's a table for the possible initial, medial and final letters: 

\begin{minipage}[t]{0.45\textwidth}
    \vspace{0.5em}
    \begin{tcolorbox}[box=Initials 초성]
        \begin{tabularx}
            {\textwidth}
            {   >{\centering\arraybackslash}X
                >{\centering\arraybackslash}X
                >{\centering\arraybackslash}X
                >{\centering\arraybackslash}X
            }
            \textbf{한글} & \textbf{Latin} & \textbf{IPA} & \textbf{Key}    \\
            \hline
            ㄱ  & g              & \ipa{k}      & \keys{D}        \\
            ㄲ  & kk             & \ipa{k͈}      & \keys{\shift D} \\
            ㄴ  & n              & \ipa{n}      & \keys{S}        \\
            ㄷ  & d              & \ipa{t}      & \keys{E}        \\
            ㄸ  & tt             & \ipa{t͈}      & \keys{\shift E} \\
            ㄹ  & r              & \ipa{ɾ}      & \keys{F}        \\
            ㅁ  & m              & \ipa{m}      & \keys{A}        \\
            ㅂ  & b              & \ipa{p}      & \keys{Q}        \\
            ㅃ  & pp             & \ipa{p͈}      & \keys{\shift Q} \\
            ㅅ  & s              & \ipa{s}      & \keys{T}        \\
            ㅆ  & ss             & \ipa{s͈}      & \keys{\shift T} \\
            ㅇ  &                &              & \keys{X}        \\
            ㅈ  & j              & \ipa{tɕ}     & \keys{W}        \\
            ㅉ  & jj             & \ipa{tɕ͈}     & \keys{\shift W} \\
            ㅊ  & ch             & \ipa{tɕʰ}    & \keys{C}        \\
            ㅋ  & k              & \ipa{kʰ}     & \keys{Z}        \\
            ㅌ  & t              & \ipa{tʰ}     & \keys{X}        \\
            ㅍ  & p              & \ipa{pʰ}     & \keys{V}        \\
            ㅎ  & h              & \ipa{h}      & \keys{G}        \\
        \end{tabularx}
    \end{tcolorbox}
\end{minipage}%
\hfill
\begin{minipage}[t]{0.45\textwidth}
    \vspace{0.5em}
    \begin{tcolorbox}[box=Medials 중성]
        \begin{tabularx}
            {\textwidth}
            {   >{\centering\arraybackslash}X
                >{\centering\arraybackslash}X
                >{\centering\arraybackslash}X
                >{\centering\arraybackslash}X
            }
            \textbf{한글} & \textbf{Latin} & \textbf{IPA} & \textbf{Key} \\
            \hline
            ㅏ  & a              & \ipa{a}      & \keys{K}          \\
            ㅐ  & ae             & \ipa{ɛ}      & \keys{O}          \\
            ㅑ  & ya             & \ipa{ja}     & \keys{I}          \\
            ㅒ  & yae            & \ipa{jɛ}     & \keys{\shift O}   \\
            ㅓ  & eo             & \ipa{ʌ}      & \keys{J}          \\
            ㅔ  & e              & \ipa{e}      & \keys{P}          \\
            ㅕ  & yeo            & \ipa{jʌ}     & \keys{U}          \\
            ㅖ  & ye             & \ipa{je}     & \keys{\shift P}   \\
            ㅗ  & o              & \ipa{o}      & \keys{H}          \\
            ㅘ  & wa             & \ipa{wa}     &                   \\
            ㅙ  & wae            & \ipa{wɛ}     &                   \\
            ㅚ  & oe             & \ipa{ø}      &                   \\
            ㅛ  & yo             & \ipa{jo}     & \keys{Y}          \\
            ㅜ  & u              & \ipa{u}      & \keys{N}          \\
            ㅝ  & wo             & \ipa{wo}     &                   \\
            ㅞ  & we             & \ipa{wɛ}     &                   \\
            ㅟ  & wi             & \ipa{wi}     &                   \\
            ㅠ  & yu             & \ipa{ju}     & \keys{B}          \\
            ㅡ  & eu             & \ipa{ɯ}      & \keys{M}          \\
            ㅢ  & ui             & \ipa{ɯi}     &                   \\
            ㅣ  & i              & \ipa{i}      & \keys{L}          \\
        \end{tabularx}
    \end{tcolorbox}
\end{minipage}
\hfill

\begin{tcolorbox}[box=Finals 종성 \textbf{(Optional)}]
    \noindent
    \begin{minipage}[t]{0.48\textwidth}
        \centering
        \textbf{Normal Finals}
        \vspace{0.5em}

        \begin{tabularx}{0.95\textwidth}{
                >{\centering\arraybackslash}X
                >{\centering\arraybackslash}X
                >{\centering\arraybackslash}X}
            \textbf{한글} & \textbf{Latin} & \textbf{IPA} \\
            \hline
            ㄱ  & g              & \ipa{k̚}      \\
            ㄴ  & n              & \ipa{n}      \\
            ㄷ  & d              & \ipa{t̚}      \\
            ㄹ  & l              & \ipa{l}      \\
            ㅁ  & m              & \ipa{m}      \\
            ㅂ  & b              & \ipa{p̚}      \\
            ㅅ  & s              & \ipa{t̚}      \\
            ㅇ  & ng             & \ipa{ŋ}      \\
            ㅈ  & j              & \ipa{t̚}      \\
            ㅊ  & ch             & \ipa{t̚}      \\
            ㅋ  & k              & \ipa{k̚}      \\
            ㅌ  & t              & \ipa{t̚}      \\
            ㅍ  & p              & \ipa{p̚}      \\
            ㅎ  & h              & \ipa{t̚}      \\
        \end{tabularx}
    \end{minipage}
    \hfill
    \begin{minipage}[t]{0.48\textwidth}
        \centering
        \textbf{Double/Complex Finals}
        \vspace{0.5em}

        \begin{tabularx}{0.95\textwidth}{
                >{\centering\arraybackslash}X
                >{\centering\arraybackslash}X
                >{\centering\arraybackslash}X}
            \textbf{한글} & \textbf{Latin} & \textbf{IPA} \\
            \hline
            ㄲ  & kk             & \ipa{k̚}      \\
            ㄳ  & gs             & \ipa{k̚}      \\
            ㄵ  & nj             & \ipa{n}      \\
            ㄶ  & nh             & \ipa{n}      \\
            ㄺ  & lg             & \ipa{k̚}      \\
            ㄻ  & lm             & \ipa{m}      \\
            ㄼ  & lb             & \ipa{p̚}      \\
            ㄽ  & ls             & \ipa{t̚}      \\
            ㄾ  & lt             & \ipa{t̚}      \\
            ㄿ  & lp             & \ipa{p̚}      \\
            ㅄ  & bs             & \ipa{p̚}      \\
            ㅆ  & ss             & \ipa{t̚}      \\
        \end{tabularx}
    \end{minipage}
\end{tcolorbox}

\subsection{Vowels}
Vowels have distinct properties that should be taken into account before trying to do anything in Korean.

\subsubsection{Bright, Dark and Neutral Vowels}
Hangul vowels follow a certain harmony which is applied when forming them

\begin{description}
    \item[Bright Vowels \rom{양성 모음}{yangseong moeum}]: ㅏ, ㅗ and ㆍ\footnote{ㆍ is an extinct character called \rom[lower a]{아래아}{araea}}
    \item[Dark Vowels \rom{음성 모음}{eumseong moeum}]: ㅓ,ㅜ and ㅡ\footnote{ㅡ is considered both partially dark and partially neutral}
    \item[Neutral Vowels \rom{중성 모음}{jungseong moeum}]: ㅣ 
\end{description}

This is really useful for when trying to study etymologies or sound symbolism, since words were often associated what type of vowel do they use\footnote{Fun fact: \rom[man]{남자}{namja} uses bright vowels and \rom[woman]{여자}{yeoja} uses dark vowels}.

Here we can see how compound monophthongs are formed:

\begin{multicols}{3}
    \begin{enumerate}
        \item [ㅐ] = ㅏ + ㅣ (Bright)
        \item [ㅔ] = ㅓ + ㅣ (Dark)
        \item [ㅒ] = ㅣ + ㅒ (Bright)
        \item [ㅖ] = | + ㅔ (Dark)
    \end{enumerate}
\end{multicols}

Then, based on these "basic vowels", we have the rest of the diphthongs as a combination of bright+bright or dark+dark:
\begin{multicols}{3}{
    \begin{enumerate}
    \item [ㅛ] = | + ㅗ (Bright)
    \item [ㅕ] = ㅣ + ㅓ (Dark)
    \item [ㅑ] = ㅣ + ㅏ (Bright)
    \item [ㅠ] = ㅣ + ㅜ (Dark)
    \item [ㅘ] = ㅗ + ㅏ (Bright)
    \item [ㅙ] = ㅗ + ㅐ (Bright)
    \item [ㅚ] = ㅗ + ㅣ (Bright)
    \item [ㅝ] = ㅜ + ㅓ (Dark)
    \item [ㅞ] = ㅜ + ㅔ (Dark)
    \item [ㅟ] = ㅜ + ㅣ (Dark)
    \item [ㅢ] = ㅡ + ㅣ (Dark)
    \end{enumerate}}
\end{multicols}

There are also certain vowels that are considered extinct or obsolete, but you may see them in older texts

\begin{multicols}{3}
    \begin{enumerate}
        \item [ㆎ] = ㆍ + ㅣ (Bright)
        \item [ㆉ] = ㅛ + ㅣ (Bright)
        \item [ㆌ] = ㅠ + ㅣ (Dark)
    \end{enumerate}
\end{multicols}

This text will use this concept when talking about grammatical rules, although it is mostly an interesting fact about how Korean phonology is formed, it is absolutely optional to learn about this and I don't think Korean schools even teach about this concept.
\section{Grammar}

Now that we know how to read Korean text, we will need to learn how to process a Korean phrase, which is why we need to understand some linguistical definitions beforehand

% TODO
\section{Speech Levels}
All verbs conjugations in the Korean language have distinct paradigms depending on the level of formality \textit{(informal vs formal)} and politeness \textit{(low vs high)} towards the listener, which are the following:

\paragraph{Higher Levels}
\begin{description}
    \item[\rom{하소서체}{hasoseo-che}]\footnote{\textbf{Etymology}\rom{체}{che} Comes from the word \textit{style}, \textit{form} or \textit{body} in Sino-Korean.}\footnote{\textbf{Etymology} \rom{하}{ha} Comes from the non-honorific imperative form of the verb \textit{to do} \rom{하다}{hada}} Very formally polite, used to address royalty or in religious texts.
    \item[\rom{하십시오체}{hasipsio-che}] Formally polite, used to address colleagues in formal settings or between strangers at the start of a conversation.
\end{description}

\paragraph{Middle levels}
\begin{description}
    \item [\rom{해요체}{haeyo-che}] Casually polite, used between strangers and colleagues.
    \item [\rom{하오체}{hao-che}] Formally neutral, used in signs or among civil servants and the older generation.
    \item [\rom{하게체}{hage-che}] Neutral, used for those under one's authority.
\end{description}

\paragraph{Lower levels}
\begin{description}
    \item [\rom{해라체}{haera-che}] Formally impolite, used with close friends or relatives and by adults to children.
    \item [\rom{해체}{hae-che}] Casually impolite or intimate, Between close friends and relatives.
\end{description}

However, there are a lot of levels that are either archaic or contextually limited, so this text will be focusing only on: \rom{해체}{hae-che}, \rom{해라체}{haera-che}, \rom{해요체}{haeyo-che}, \rom{하십시오체}{hasipsio-che} and will be called \textit{informal low}, \textit{formal low}, \textit{informal high} and \textit{formal high} respectively for convenience and because they cover 99\% of real-life usage.
\section*{Before we continue\dots}

It is worth warning that I'm not a linguistics professor nor did I study academicaly this subject. These are notes that I've taken while I was learning the language by my own accounts. If you're reading this, they somehow worked for me and I unfortunately wanted to share my monstruosity to the public.

This text has the intention of being read in any order, so there might be repeated information. However, in order to avoid redundancy, the text will assume you've read and understood this chapter and will not repeat any information explained in said chapter.

Lastly, It is worth warning that this text will try to use the least amount of words possible. This is due to the fact that adding new unnecessary vocabulary may result overwhelming for the learning process of a specific section. 



\chapter{Verbs}
\section{Speech Levels}
All verbs conjugations in the Korean language have distinct paradigms depending on the level of formality \textit{(informal vs formal)} and politeness \textit{(low vs high)} towards the listener, which are the following:

\paragraph{Higher Levels}
\begin{description}
    \item[\rom{하소서체}{hasoseo-che}]\footnote{\textbf{Etymology}\rom{체}{che} Comes from the word \textit{style}, \textit{form} or \textit{body} in Sino-Korean.}\footnote{\textbf{Etymology} \rom{하}{ha} Comes from the non-honorific imperative form of the verb \textit{to do} \rom{하다}{hada}} Very formally polite, used to address royalty or in religious texts.
    \item[\rom{하십시오체}{hasipsio-che}] Formally polite, used to address colleagues in formal settings or between strangers at the start of a conversation.
\end{description}

\paragraph{Middle levels}
\begin{description}
    \item [\rom{해요체}{haeyo-che}] Casually polite, used between strangers and colleagues.
    \item [\rom{하오체}{hao-che}] Formally neutral, used in signs or among civil servants and the older generation.
    \item [\rom{하게체}{hage-che}] Neutral, used for those under one's authority.
\end{description}

\paragraph{Lower levels}
\begin{description}
    \item [\rom{해라체}{haera-che}] Formally impolite, used with close friends or relatives and by adults to children.
    \item [\rom{해체}{hae-che}] Casually impolite or intimate, Between close friends and relatives.
\end{description}

However, there are a lot of levels that are either archaic or contextually limited, so this text will be focusing only on: \rom{해체}{hae-che}, \rom{해라체}{haera-che}, \rom{해요체}{haeyo-che}, \rom{하십시오체}{hasipsio-che} and will be called \textit{informal low}, \textit{formal low}, \textit{informal high} and \textit{formal high} respectively for convenience. This will only be for verb conjugations since the four levels previously mentioned cover 99\% of real-life usage.

Also it is worth warning that this text will try to use the least amount of verbs possible. It is recommended to use a dictionary, texts or conversations in order to expand one's vocabulary. This text should be used in order to learn how to structure said vocabulary.
\section{Endings}
\subsection{Grammatical Mood Endings}
Depending on the grammatical mood \rom[form]{형}{hyeong} of the sentence, each level has a specific ending. The four most important grammatical moods and the ones explained in further detail are \textit{indicative, interrogative, imperative and propositive}.

\begin{tcolorbox}[box=Imperative 명령형]
\begin{tabular}{c|>{\centering\arraybackslash}m{2.5cm}|>{\centering\arraybackslash}m{\dimexpr0.3\linewidth}|>{\arraybackslash}m{5cm}}
    \textbf{Level} & \textbf{Ending} & \textbf{Rule} (ends with) & \textbf{Examples} (present)
    \\\hline
    \textbf{IL} 
        & \shortstack{--아라 \\ --어라}
        & \shortstack{bright vowel $\rightarrow$ --아라 \\ 
                      otherwise $\rightarrow$ --어라} 
        & \spacedstack{
            \rom[]{\textbf{\color{magenta}가}\color{blue}다}{\textbf{\color{magenta}ga}\color{blue}da}$\rightarrow$\rom[]{가라}{gala}\\
            \rom[]{\textbf{\color{violet}먹}\color{blue}다}{\textbf{\color{violet}meog}\color{blue}da}$\rightarrow$\rom[]{먹어라}{meogeola}
        }
    \\\hline
    \textbf{IH}
        & \shortstack{--세요 \\ --으세요 }
        & \shortstack{bright vowel $\rightarrow$ --세요 \\ 
                      otherwise $\rightarrow$ --으세요} 
        & \spacedstack{
            \rom[]{\textbf{\color{magenta}가}\color{blue}다}{\textbf{\color{magenta}ga}\color{blue}da}$\rightarrow$\rom[]{가세요}{gaseyo}\\
            \rom[]{\textbf{\color{violet}먹}\color{blue}다}{\textbf{\color{violet}meog}\color{blue}da}$\rightarrow$\rom[]{먹으세요}{meogeuseyo}
        }
    \\\hline
    \textbf{FL}\footnote{Note that, in this grammatical mood, it sounds exactly the same as in \textit{informal low}}
        & \shortstack{--아라 \\ --어라}
        & \shortstack{vowel $\rightarrow$ --아라 \\ 
                      consonant $\rightarrow$ --어라} 
        & \spacedstack{
            \rom[]{\textbf{\color{magenta}가}\color{blue}다}{\textbf{\color{magenta}ga}\color{blue}da}$\rightarrow$\rom[]{가라}{gala}\\
            \rom[]{\textbf{\color{violet}먹}\color{blue}다}{\textbf{\color{violet}meog}\color{blue}da}$\rightarrow$\rom[]{먹어라}{meogeola}
        }
    \\\hline
    \textbf{FH}
        & \shortstack{--십시오 \\ --으십시오 }
        & \shortstack{vowel $\rightarrow$ --십시오 \\ 
                      consonant $\rightarrow$ --으십시오} 
        & \spacedstack{
            \rom[]{\textbf{\color{magenta}가}\color{blue}다}{\textbf{\color{magenta}ga}\color{blue}da}$\rightarrow$\rom[]{가십시오}{gasibsio}\\
            \rom[]{\textbf{\color{violet}먹}\color{blue}다}{\textbf{\color{violet}meog}\color{blue}da}$\rightarrow$\rom[]{먹으십시오}{meogeosibsio}
        }
\end{tabular}
\end{tcolorbox}
\begin{tcolorbox}[box=Interrogative 의문형]
\begin{tabular}{c|>{\centering\arraybackslash}m{2.5cm}|>{\centering\arraybackslash}m{\dimexpr0.3\linewidth}|>{\arraybackslash}m{5cm}}
    \textbf{Level} & \textbf{Ending} & \textbf{Rule} (ends with) & \textbf{Examples} (present)
    \\\hline
    \textbf{IL} 
        & \shortstack{--니? \\ --어?}
        & \shortstack{bright vowel $\rightarrow$ --니? \\ 
                      otherwise $\rightarrow$ --어?} 
        & \spacedstack{
            \rom[]{\textbf{\color{magenta}가}\color{blue}다}{\textbf{\color{magenta}ga}\color{blue}da}$\rightarrow$\rom[]{가니?}{gani?}\\
            \rom[]{\textbf{\color{violet}먹}\color{blue}다}{\textbf{\color{violet}meog}\color{blue}da}$\rightarrow$\rom[]{먹어?}{meogeo?}
        }
    \\\hline
    \textbf{IH}
        & \shortstack{--아요? \\ --어요? }
        & \shortstack{bright vowel $\rightarrow$ --아요? \\ 
                      otherwise $\rightarrow$ --어요?} 
        & \spacedstack{
            \rom[]{\textbf{\color{magenta}가}\color{blue}다}{\textbf{\color{magenta}ga}\color{blue}da}$\rightarrow$\rom[]{가요?}{gayo?}\\
            \rom[]{\textbf{\color{violet}먹}\color{blue}다}{\textbf{\color{violet}meog}\color{blue}da}$\rightarrow$\rom[]{먹어요?}{meogeoyo?}
        }
    \\\hline
    \textbf{FL}
        & \shortstack{--느냐? \\ --냐?}
        & \shortstack{vowel $\rightarrow$ --느냐? \\ 
                      consonant $\rightarrow$ --냐?} 
        & \spacedstack{
            \rom[]{\textbf{\color{magenta}가}\color{blue}다}{\textbf{\color{magenta}ga}\color{blue}da}$\rightarrow$\rom[]{가냐?}{ganya?}\\
            \rom[]{\textbf{\color{violet}먹}\color{blue}다}{\textbf{\color{violet}meog}\color{blue}da}$\rightarrow$\rom[]{먹느냐?}{meogneunya}
        }
    \\\hline
    \textbf{FH}
        & \shortstack{--ㅂ니까? \\ --습니까? }
        & \shortstack{vowel $\rightarrow$ --ㅂ니까? \\ 
                      consonant $\rightarrow$ --습니까?} 
        & \spacedstack{
            \rom[]{\textbf{\color{magenta}가}\color{blue}다}{\textbf{\color{magenta}ga}\color{blue}da}$\rightarrow$\rom[]{갑니까?}{gabnikka}\\
            \rom[]{\textbf{\color{violet}먹}\color{blue}다}{\textbf{\color{violet}meog}\color{blue}da}$\rightarrow$\rom[]{먹습니까?}{meogseubnikka?}
        }
\end{tabular}
\end{tcolorbox}
\begin{tcolorbox}[box=Imperative 명령형]
\begin{tabular}{c|>{\centering\arraybackslash}m{2.5cm}|>{\centering\arraybackslash}m{\dimexpr0.3\linewidth}|>{\arraybackslash}m{5cm}}
    \textbf{Level} & \textbf{Ending} & \textbf{Rule} (ends with) & \textbf{Examples} (present)
    \\\hline
    \textbf{IL} 
        & \shortstack{--아라 \\ --어라}
        & \shortstack{bright vowel $\rightarrow$ --아라 \\ 
                      otherwise $\rightarrow$ --어라} 
        & \spacedstack{
            \rom[]{\textbf{\color{magenta}가}\color{blue}다}{\textbf{\color{magenta}ga}\color{blue}da}$\rightarrow$\rom[]{가라}{gala}\\
            \rom[]{\textbf{\color{violet}먹}\color{blue}다}{\textbf{\color{violet}meog}\color{blue}da}$\rightarrow$\rom[]{먹어라}{meogeola}
        }
    \\\hline
    \textbf{IH}
        & \shortstack{--세요 \\ --으세요 }
        & \shortstack{bright vowel $\rightarrow$ --세요 \\ 
                      otherwise $\rightarrow$ --으세요} 
        & \spacedstack{
            \rom[]{\textbf{\color{magenta}가}\color{blue}다}{\textbf{\color{magenta}ga}\color{blue}da}$\rightarrow$\rom[]{가세요}{gaseyo}\\
            \rom[]{\textbf{\color{violet}먹}\color{blue}다}{\textbf{\color{violet}meog}\color{blue}da}$\rightarrow$\rom[]{먹으세요}{meogeuseyo}
        }
    \\\hline
    \textbf{FL}\footnote{Note that, in this grammatical mood, it sounds exactly the same as in \textit{informal low}}
        & \shortstack{--아라 \\ --어라}
        & \shortstack{vowel $\rightarrow$ --아라 \\ 
                      consonant $\rightarrow$ --어라} 
        & \spacedstack{
            \rom[]{\textbf{\color{magenta}가}\color{blue}다}{\textbf{\color{magenta}ga}\color{blue}da}$\rightarrow$\rom[]{가라}{gala}\\
            \rom[]{\textbf{\color{violet}먹}\color{blue}다}{\textbf{\color{violet}meog}\color{blue}da}$\rightarrow$\rom[]{먹어라}{meogeola}
        }
    \\\hline
    \textbf{FH}
        & \shortstack{--십시오 \\ --으십시오 }
        & \shortstack{vowel $\rightarrow$ --십시오 \\ 
                      consonant $\rightarrow$ --으십시오} 
        & \spacedstack{
            \rom[]{\textbf{\color{magenta}가}\color{blue}다}{\textbf{\color{magenta}ga}\color{blue}da}$\rightarrow$\rom[]{가십시오}{gasibsio}\\
            \rom[]{\textbf{\color{violet}먹}\color{blue}다}{\textbf{\color{violet}meog}\color{blue}da}$\rightarrow$\rom[]{먹으십시오}{meogeosibsio}
        }
\end{tabular}
\end{tcolorbox}
\begin{tcolorbox}[box=Propositive 청유형]
\begin{tabular}{c|>{\centering\arraybackslash}m{2.5cm}|>{\centering\arraybackslash}m{\dimexpr0.3\linewidth}|>{\arraybackslash}m{5cm}}
    \textbf{Level} & \textbf{Ending} & \textbf{Rule} (ends with) & \textbf{Examples} (present)
    \\\hline
    \textbf{IL} 
        & \shortstack{--자}
        & \shortstack{} 
        & \spacedstack{
            \rom[]{\textbf{\color{magenta}가}\color{blue}다}{\textbf{\color{magenta}ga}\color{blue}da}$\rightarrow$\rom[]{가자}{gaja}\\
            \rom[]{\textbf{\color{violet}먹}\color{blue}다}{\textbf{\color{violet}meog}\color{blue}da}$\rightarrow$\rom[]{먹자}{meogja}
        }
    \\\hline
    \textbf{IH}
        & \shortstack{--아요 \\ --어요 }
        & \shortstack{bright vowel $\rightarrow$ --아요 \\ 
                      otherwise $\rightarrow$ --어요} 
        & \spacedstack{
            \rom[]{\textbf{\color{magenta}가}\color{blue}다}{\textbf{\color{magenta}ga}\color{blue}da}$\rightarrow$\rom[]{가요}{gayo}\\
            \rom[]{\textbf{\color{violet}먹}\color{blue}다}{\textbf{\color{violet}meog}\color{blue}da}$\rightarrow$\rom[]{먹어요}{meogeoyo}
        }
    \\\hline
    \textbf{FL}
        & \shortstack{--게}
        & \shortstack{} 
        & \spacedstack{
            \rom[]{\textbf{\color{magenta}가}\color{blue}다}{\textbf{\color{magenta}ga}\color{blue}da}$\rightarrow$\rom[]{가게}{gage}\\
            \rom[]{\textbf{\color{violet}먹}\color{blue}다}{\textbf{\color{violet}meog}\color{blue}da}$\rightarrow$\rom[]{먹게}{meogge}
        }
    \\\hline
    \textbf{FH}
        & \shortstack{--ㅂ시다 \\ --읍시다 }
        & \shortstack{vowel $\rightarrow$ --ㅂ시다 \\ 
                      consonant $\rightarrow$ --읍시다} 
        & \spacedstack{
            \rom[]{\textbf{\color{magenta}가}\color{blue}다}{\textbf{\color{magenta}ga}\color{blue}da}$\rightarrow$\rom[]{갑시다}{gabnida}\\
            \rom[]{\textbf{\color{violet}먹}\color{blue}다}{\textbf{\color{violet}meog}\color{blue}da}$\rightarrow$\rom[]{먹읍시다}{meogeubnida}
        }
\end{tabular}
\end{tcolorbox}
\section{Grammatical Tenses}
\subsection{Declarative Present Tense (평서형현재시제)}

This grammatical tense is used to \textbf{state facts, describe habits} or \textbf{give information}

\begin{description}
    \item[Informal Low]: --아 (if ends with a bright vowel) / --어 (if anything else)
    \item[Informal High]: --아요 (if ends with a bright vowel) / --어요 (if anything else)
    \item[Formal Low]: --는다 (if ends with a consonant) / --ㄴ다 (if ends with a vowel)
    \item[Formal High] : --습니다  (if ends with a consonant) / --ㅂ니다 (if ends with a vowel) 
\end{description}

Here are some examples:

\begin{center}
    \begin{tabular}{c|c|c|c|c}
        \textbf{Verb Stem} & \textbf{Informal Low} & \textbf{Informal High} & \textbf{Formal Low} & \textbf{Formal High}\\
        \hline
        
        \rom[to go]{\textbf{\color{magenta}가}\color{blue}다}{\textbf{\color{magenta}ga}\color{blue}da}             
            & \rom{가}{ga}
            & \rom{가요}{gayo}
            & \rom{간다}{ganda}
            & \rom{갑니다}{gabnida} \\
        \hline

        \rom[to eat]{\textbf{\color{magenta}먹}\color{blue}다}{\textbf{\color{magenta}meog}\color{blue}da} 
            & \rom{먹어}{meogeo}
            & \rom{먹어요}{meogeoyo}
            & \rom{먹는다}{meogneunda}
            & \rom{먹습니다}{meogseubnida} \\
        \hline

        \rom[to drink]{마\textbf{\color{magenta}시}\color{blue}다}{ma\textbf{\color{magenta}sa}\color{blue}da}
            & \rom{마셔}{masyeo}
            & \rom{마셔요}{masyeoyo}
            & \rom{마신다}{masinda}
            & \rom{마십니다}{masibnida}

    \end{tabular}
\end{center}

Korean has several groups of verbs that don't follow these standard conjugation rules, and there are certain verbs that should count as irregular but are not for some reason in particular, it is advised to take these groups with a grain of salt.

Here's a quick cheat sheet for all the groups:

\begin{tabularx}{\linewidth}{X|X}
    \textbf{Vowel after ㄷ--ending stem}
        \begin{itemize}
            \item [Rule]: ㄷ $\rightarrow$ ㄹ
            \item [E.g]: \rom[to walk]{걷다}{geodda}
        \end{itemize}
        \begin{tabular}{c|c|c|c}
            \textbf{IL} & \textbf{IH} & \textbf{FL} & \textbf{FH} \\
            \rom{걸어}{geoleo} & \rom{걸어요}{geoleoyo} & \rom{걷는다}{geodneunda} & \rom{걷습니다}{geodseubnida} 
        \end{tabular}
    & \textbf{Vowel after ㅂ--ending stem}
        \begin{itemize}
            \item [Rule]: ㅂ $\rightarrow$ 우/오\footnote{Left if the verb has dark vowels, right if bright.} (Only in IL/IH)\footnote{It was previously used in Formal High as well, so some dictionaries may accept both as valid}
            \item [E.g]: \rom[to help]{돕다}{dobda}
        \end{itemize}
        \begin{tabular}{c|c|c|c}
            \textbf{IL} & \textbf{IH} & \textbf{FL} & \textbf{FH} \\
            \rom{도와}{dowa} & \rom{도와요}{dowayo} & \rom{돕는다}{dobneunda} & \rom{돕습니다}{dobseubnida} 
        \end{tabular}
    \\ \hline
    \textbf{아 / 어요 after 르 stem}
        \begin{itemize}
            \item [Rule]: Add ㄹ to last syllable
            \item [E.g]: \rom[to differ]{다르다}{daleuda}
        \end{itemize}
        \begin{tabular}{c|c|c|c}
            \textbf{IL} & \textbf{IH} & \textbf{FL} & \textbf{FH} \\
            \rom{달라}{dalla} & \rom{달라요}{dallayo} & \rom{다른다}{daleunda} & \rom{다릅니다}{daleubnida} 
        \end{tabular}
    & \textbf{Vowel after ㅅ--ending stem}
        \begin{itemize}
            \item [Rule]: Drop ㅅ
            \item [E.g]: \rom[to surpass]{낫다}{nasda}
        \end{itemize}
        \begin{tabular}{c|c|c|c}
            \textbf{IL} & \textbf{IH} & \textbf{FL} & \textbf{FH} \\
            \rom{나아}{naa} & \rom{나아요}{naayo} & \rom{낫는다}{nasneunda} & \rom{낫습니다}{nasseubnida}
        \end{tabular}
    \\ \hline
    \textbf{Stems ending in ㅎ\footnote{Usually adjectives or \textit{descriptive verbs}}}
        \begin{itemize}
            \item [Rule]: Drop ㅎ, Use ㅐ (Only in IL/IH)
            \item [E.g]: \rom[to be white]{하얗다}{hayahda}
        \end{itemize}
        \begin{tabular}{c|c|c|c}
            \textbf{IL} & \textbf{IH} & \textbf{FL} & \textbf{FH} \\
            \rom{하얘}{hayae} & \rom{하얘요}{hayaeyo} & \rom{하얗다}{hayahneunda} & \rom{하얗습니다}{hayahseubnida} 
        \end{tabular}
    & \textbf{Stems ending in ㅡ}
        \begin{itemize}
            \item [Rule]: Drop ㅡ, vowel merge (Only in IL/IH)
            \item [E.g]: \rom[to write]{쓰다}{sseuda}
        \end{itemize}
        \begin{tabular}{c|c|c|c}
            \textbf{IL} & \textbf{IH} & \textbf{FL} & \textbf{FH} \\
            \rom{써}{sseo} & \rom{써요}{sseoyo} & \rom{쓴다}{sseunda} & \rom{씁니다}{sseubnida} 
        \end{tabular}
\end{tabularx}

\subsection{Declarative Past Tense (평서과거시제)}

\begin{description}
    \item[Informal Low]: --았어 (if ends with a bright vowel) / --었어 (if anything else)
    \item[Informal High]: --았아요 (if ends with a bright vowel) / --었어요 (if anything else)
    \item[Formal Low]: --았다 (if ends with a consonant) / --었다 (if ends with a vowel)
    \item[Formal High] : --았습니다  (if ends with a consonant) / --었습니다 (if ends with a vowel) 
\end{description}

\begin{center}
    \begin{tabular}{c|c|c|c|c}
        \textbf{Verb Stem} & \textbf{Informal Low} & \textbf{Informal High} & \textbf{Formal Low} & \textbf{Formal High}\\
        \hline
        
        \rom[to go]{\textbf{\color{magenta}가}\color{blue}다}{\textbf{\color{magenta}ga}\color{blue}da}             
            & \rom{갔어}{ga}
            & \rom{갔어요}{gayo}
            & \rom{갔다}{ganda}
            & \rom{갔습니다}{gabnida} \\
        \hline

        \rom[to eat]{\textbf{\color{magenta}먹}\color{blue}다}{\textbf{\color{magenta}meog}\color{blue}da} 
            & \rom{먹어}{meogeo}
            & \rom{먹어요}{meogeoyo}
            & \rom{먹는다}{meogneunda}
            & \rom{먹습니다}{meogseubnida} \\
        \hline

        \rom[to drink]{마\textbf{\color{magenta}시}\color{blue}다}{ma\textbf{\color{magenta}sa}\color{blue}da}
            & \rom{마셔}{masyeo}
            & \rom{마셔요}{masyeoyo}
            & \rom{마신다}{masinda}
            & \rom{마십니다}{masibnida}
    \end{tabular}
\end{center}

As you can see, the syllable 갔 is present regardless of the speech level, this is what we will call \textit{pre-final ending}

\chapter{Functional Categories}
\section{Pronouns}
Pronouns are words that replace nouns in a sentence. In Korean, they can be divided into three groups: \textbf{personal}, \textbf{demonstrative} and \textbf{interrogative}.

\begin{itemize}
    \item \textbf{Personal pronouns} are words that are associated to a specific grammatical person (\textit{I, you, they, he, etc.}). 
    
    These words can be categorized in two sections: the person (\textit{first, second or third}) and the number (\textit{singular or plural}), in which the latter can be optionally subdivided into grammatical gender (masculine, femenine or object). It is worth noting that native speakers generally avoid third-person pronouns and are rarely used in spoken conversation.
    
    In Korean, personal pronouns are divided between \textbf{informal} and \textbf{polite or formal} context.

    \begin{center}
        \begin{tcolorbox}[box=Personal Pronouns]
            \begin{center}
                \begin{tabular}{cc|c|c}
                    \textbf{Person} & \textbf{Number} & \textbf{Informal} & \textbf{Formal} \\
                    \hline
                    1st & Singular & \rom[I]{나}{na} & \rom{저}{jeo} \\
                    \hline
                    2nd & Singular & \rom[You]{너}{neo} & \rom[]{그쪽}{geujjok} \\
                    \hline
                    3rd & Singular (General/Masculine) & \rom[He]{그}{geu} & \rom{그분}{geubun} \\
                    \hline
                    & Singular (Femenine)\footnote{This word was created in order to translate third person pronouns from foreign countries, again, third-person pronouns are rarely used in spoken conversation} & \rom[She]{그녀}{geunyeo} & \rom{그분}{geubun} \\
                    \hline
                    & Singular (Object) & \rom[It]{그것}{geugeot} \\
                    \hline
                    1st & Plural & \rom[We]{우리}{uri} & \rom{저희}{jeohui} \\
                    \hline
                    2nd & Plural & \rom[You]{너희}{neohui} & \rom{여러분}{yeoreobun} \\
                    \hline
                    3rd & Plural & \rom[They]{그들}{geudeul} & \rom{그분들}{geubundeul} \\
                \end{tabular}
            \end{center}
        \end{tcolorbox}
    \end{center}
    
    \item \textbf{Demonstrative pronouns} are words that point to specific things\footnote{\textbf{Etymology} \rom[]{것}{geot} comes from the word \textit{thing}, which is used in the third-person singular pronoun for inanimate objects and all of the demonstrative pronouns.} (\textit{this, that, these or those}). 
    
    These words can be categorized in two sections: the distance (\textit{proximal, medial or distal}) and the number (\textit{singular or plural}). 
    
    Note that in English there aren't any differences between medial and distal demosntrative pronouns, although one could accentuate the distance by adding "over there" to the pronoun (\textit{that} versus \textit{that over there}).
    
    \begin{center}
        \begin{minipage}[t]{0.7\textwidth}
        \begin{tcolorbox}[box=Demonstrative Pronouns]
            \begin{center}
                \begin{tabular}{cc|c}
                    \textbf{Distance} & \textbf{Number} & \\
                    \hline
                    Proximal & Singular & \rom[this]{이것}{igeot} \\
                    \hline
                    Medial & Singular & \rom[that]{그것}{geugeot} \\
                    \hline
                    Distal & Singular & \rom[that]{저것}{jeogeot} \\
                    \hline
                    Proximal & PL & \rom[these]{이것들}{igeotdeul}\footnote{\textbf{Note} \rom[]{들}{deul} is a \textit{plural marker}} \\
                    \hline
                    Medial & Plural & \rom[those]{그것들}{geugeotdeul} \\
                    \hline
                    Distal & Plural & \rom[those]{저것들}{jeogeotdeul} \\
        
                \end{tabular}
            \end{center}
        \end{tcolorbox}
        \end{minipage}
    \end{center}
    
    \item \textbf{Interrogative pronouns} are words used to ask a question, (i.e \textit{what, which, when, etc\dots})

    \begin{center}
        \begin{minipage}[t]{0.7\textwidth}
        \begin{tcolorbox}[box=Demonstrative Pronouns]
            \begin{center}
                \begin{tabular}{c|c}
                    \textbf{Function} & \\
                    \hline
                    Person & \rom[who]{누구}{nugu} \\
                    \hline
                    Thing & \rom[what]{무엇}{mueos}/\rom[what]{뭐}{mua}\footnote{무엇 is more formal} \\
                    \hline
                    Place & \rom[where]{어디}{eodi} \\
                    \hline
                    Time & \rom[when]{언제}{eonje} \\
                \end{tabular}
                \hfill
                \begin{tabular}{c|c}
                    \textbf{Function} & \\
                    \hline
                    Reason & \rom[why]{왜}{wae} \\
                    \hline
                    Manner & \rom[how]{어떻게}{eotteohge} \\
                    \hline
                    Selection & \rom[which]{어느}{eoneu} \\
                    \hline
                    Quantity & \rom[how many]{몇}{myeoch} \\
        
                \end{tabular}
            \end{center}
        \end{tcolorbox}
        \end{minipage}
    \end{center}

\end{itemize}





\section{Particles}
Unlike English, which follows an $Subject-Verb-Object$ order, Korean follows $Subject-Object-Verb$ ordering.

Also, Korean relies heavily on particles to show each word's grammatical role, here are some examples\footnote{With the particles: use left if ends in a consonant, right if it ends in a vowel} with literal translation

\begin{center}
    \begin{tabular}{
        l|c|c|l}
    
        \textbf{Role} & \textbf{Particle} & \textbf{Example} & \textbf{Meaning} \\ \hline
    
        Subject & 
            이 / 가 & 
            \rom[I]{제\textbf{\color{magenta}가}}{je\textbf{\color{magenta}ga}} 
            \rom[apple]{사과\textbf{\color{blue}를}}{sagwa\textbf{\color{blue}reul}} 
            \rom[eat]{먹어요}{meogeoyo} &
            \textbf{\color{magenta}I} eat an \textbf{\color{blue}apple}
            \\ \hline
        Topic & 
            은 / 는 &
            \rom[I]{저\textbf{\color{magenta}는}}{je\textbf{\color{magenta}neun}} 
            \rom[apple]{사과\textbf{\color{blue}를}}{sagwa\textbf{\color{blue}reul}} 
            \rom[eat]{먹어요}{meogeoyo} &
            \textbf{\color{magenta}As for me, I} eat an \textbf{\color{blue}apple}
            \\ \hline
        Object &
            을 / 를 &
            \rom[I]{제\textbf{\color{magenta}가}}{je\textbf{\color{magenta}ga}} 
            \rom[apple]{사과\textbf{\color{blue}를}}{sagwa\textbf{\color{blue}reul}}
            \rom[eat]{먹어요}{meogeoyo} &
            \textbf{\color{magenta}I} eat an \textbf{\color{blue}apple}
            \\ \hline
        Possession &
            의 &
            \rom[I]{제\textbf{\color{magenta}가}}{je\textbf{\color{magenta}ga}}
            \rom[God's]{신\textbf{\color{orange}의}}{sin\textbf{\color{orange}ui}}
            \rom[apple]{사과\textbf{\color{blue}를}}{sagwa\textbf{\color{blue}reul}} 
            \rom[eat]{먹어요}{meogeoyo} &
            \textbf{\color{magenta}I} eat God\textbf{\color{orange}'s} \textbf{\color{blue}apple}
            \\ \hline
        Location &
            에서 &
            \rom[I]{제\textbf{\color{magenta}가}}{je\textbf{\color{magenta}ga}}
            \rom[at the garden]{정원\textbf{\color{purple}에서}}{jeongwon\textbf{\color{purple}eseo}}
            \rom[apple]{사과\textbf{\color{blue}를}}{sagwa\textbf{\color{blue}reul}}
            \rom[eat]{먹어요}{meogeoyo} &
            \textbf{\color{magenta}I} eat an \textbf{\color{blue}apple} \textbf{\color{purple}at} the garden

    \end{tabular}
\end{center}

In informal speech, these particles are often dropped if the context is clear. Also note that there are much more particles than the ones in this table, however, they will be explained in their respective sections.


\end{document}

% 1. Introduction to Korean Language
%     1.1 Overview of Hangul
%     1.2 Korean as an agglutinative language
%     1.3 Syntax and SOV structure
%     1.4 Honorifics and speech levels

% 2. Building Vocabulary – Core Lexical Categories
%     2.1 Nouns
%         2.1.1 People, places, objects
%         2.1.2 Abstract concepts
%         2.1.3 Sino-Korean vs. native nouns
%     2.2 Verbs
%         2.2.1 Basic conjugations
%         2.2.2 Verb stems and endings
%     2.3 Adjectives
%         2.3.1 Descriptive verbs in Korean
%         2.3.2 Conjugation similarities with verbs
%     2.4 Adverbs
%         2.4.1 Placement in sentences
%         2.4.2 Common adverbs (frequency, manner, time)
%     2.5 Ideophones
%         2.5.1 Sound-symbolic vocabulary
%         2.5.2 Usage in spoken Korean
%     2.6 Numerals
%         2.6.1 Native Korean numbers
%         2.6.2 Sino-Korean numbers
%         2.6.3 Counting people, objects, time

% 4. Structure and Function – Functional Categories
%     3.1 Pronouns
%         3.1.1 Personal pronouns
%         3.1.2 Demonstrative pronouns
%         3.1.3 Interrogative pronouns
%     3.2 Determiners
%         3.2.1 이, 그, 저
%         3.2.2 Quantity and specificity
%     3.3 Particles
%         3.3.1 Subject, topic, object particles
%         3.3.2 Locative, time, direction particles
%         3.3.3 Clause-connecting particles
%     3.4 Counters
%         3.4.1 Usage with numerals
%         3.4.2 Categories of counters
%     3.5 Interjections
%         3.5.1 Expressing emotion in speech
%         3.5.2 Sound expressions and responses

% 4. Expressing Ideas – Phrases and Multiword Terms
%     4.1 Common Phrases
%         4.1.1 Greetings
%         4.1.2 Daily expressions
%         4.1.3 Speech-level differences
%     4.2 Multiword Terms
%         4.2.1 Idiomatic expressions
%         4.2.2 Collocations and set phrases

% 5. Morphology – The Building Blocks
%     5.1 Morphemes
%         5.1.1 Roots and stems
%         5.1.2 Derivational suffixes
%         5.1.3 Honorific and tense markers
%     5.2 Agglutination and Verb Conjugation
%         5.2.1 Politeness levels
%         5.2.2 Tense, aspect, mood
%         5.2.3 Negative and question forms

% 6. Putting It All Together
%     6.1 Constructing sentences from words
%     6.2 Sentence particles in practice
%     6.3 Thematic practice sentences

% Appendices
%     A. Hangul chart and pronunciation guide
%     B. Common particles reference table
%     C. Common counters and their noun domains
%     D. High-frequency verbs and adjectives
%     E. Sample dialogues by speech level