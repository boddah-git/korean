\section{Pronouns}
Pronouns are words that replace nouns in a sentence. In Korean, they can be divided into two groups: \textbf{personal} and \textbf{demonstrative}.

\begin{itemize}
    \item \textbf{Personal pronouns} are words that are associated to a specific grammatical person (\textit{I, you, they, he, etc.}). These words can be categorized in two sections: the person (\textit{first, second or third}) and the number (\textit{singular or plural}). In Korean, personal pronouns are divided between \textbf{informal} and \textbf{polite or formal} context.
    
    \item \textbf{Demonstrative pronouns} are words that point to specific things\footnote{\textbf{Etymology} \rom[]{것}{geot} comes from the word \textit{thing}} (\textit{this, that, these or those}). These words can be categorized in two sections: the distance (\textit{proximal, medial or distal}) and the number (\textit{singular or plural}). Note that in English there aren't any differences between medial and distal demosntrative pronouns, although one could accentuate the distance by adding "over there" to the pronoun (\textit{that} versus \textit{that over there}).
\end{itemize}


\begin{minipage}[t]{0.6\textwidth}
\begin{tcolorbox}[box=Personal Pronouns]
    \begin{center}
        \begin{tabular}{cc|c|c}
            \textbf{Person} & \textbf{Num.} & \textbf{Informal} & \textbf{Formal} \\
            \hline
            1st & SG & \rom[I]{나}{na} & \rom{저}{jeo} \\
            \hline
            2nd & SG & \rom[You]{너}{neo} & \rom[]{그쪽}{geujjok} \\
            \hline
            3rd & SG (M) & \rom[He]{그}{geu} & \rom{그분}{geubun} \\
            \hline
            & SG (F) & \rom[She]{그녀}{geunyeo} & \rom{그분}{geubun} \\
            \hline
            & SG (O) & \rom[It]{그것}{geugeot} \\
            \hline
            1st & PL & \rom[We]{우리}{uri} & \rom{저희}{jeohui} \\
            \hline
            2nd & PL & \rom[You]{너희}{neohui} & \rom{여러분}{yeoreobun} \\
            \hline
            3rd & PL & \rom[They]{그들}{geudeul} & \rom{그분들}{geubundeul} \\
        \end{tabular}
    \end{center}
\end{tcolorbox}
\end{minipage}
\hfill
\begin{minipage}[t]{0.4\textwidth}
\begin{tcolorbox}[box=Demonstrative Pronouns]
    \begin{tabular}{cc|c}
        \textbf{Dist.} & \textbf{Num.} & \\
        \hline
        Prox. & SG & \rom[this]{이것}{igeot} \\
        \hline
        Med. & SG & \rom[that]{그것}{geugeot} \\
        \hline
        Dist. & SG & \rom[that]{저것}{jeogeot} \\
        \hline
        Prox. & PL & \rom[these]{이것들}{igeotdeul}\footnote{\rom[]{들}{deul} is a \textit{plural marker}} \\
        \hline
        Med. & PL & \rom[those]{그것들}{geugeotdeul} \\
        \hline
        Dist. & PL & \rom[those]{저것들}{jeogeotdeul} \\

    \end{tabular}
\end{tcolorbox}
\end{minipage}