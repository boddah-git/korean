\section{Particles}
Unlike English, which follows an $Subject-Verb-Object$ order, Korean follows $Subject-Object-Verb$ ordering.

Also, Korean relies heavily on particles to show each word's grammatical role, here are some examples\footnote{With the particles: use left if ends in a consonant, right if it ends in a vowel} with literal translation

\begin{center}
    \begin{tabular}{
        l|c|c|l}
    
        \textbf{Role} & \textbf{Particle} & \textbf{Example} & \textbf{Meaning} \\ \hline
    
        Subject & 
            이 / 가 & 
            \rom[I]{제\textbf{\color{magenta}가}}{je\textbf{\color{magenta}ga}} 
            \rom[apple]{사과\textbf{\color{blue}를}}{sagwa\textbf{\color{blue}reul}} 
            \rom[eat]{먹어요}{meogeoyo} &
            \textbf{\color{magenta}I} eat an \textbf{\color{blue}apple}
            \\ \hline
        Topic & 
            은 / 는 &
            \rom[I]{저\textbf{\color{magenta}는}}{je\textbf{\color{magenta}neun}} 
            \rom[apple]{사과\textbf{\color{blue}를}}{sagwa\textbf{\color{blue}reul}} 
            \rom[eat]{먹어요}{meogeoyo} &
            \textbf{\color{magenta}As for me, I} eat an \textbf{\color{blue}apple}
            \\ \hline
        Object &
            을 / 를 &
            \rom[I]{제\textbf{\color{magenta}가}}{je\textbf{\color{magenta}ga}} 
            \rom[apple]{사과\textbf{\color{blue}를}}{sagwa\textbf{\color{blue}reul}}
            \rom[eat]{먹어요}{meogeoyo} &
            \textbf{\color{magenta}I} eat an \textbf{\color{blue}apple}
            \\ \hline
        Possession &
            의 &
            \rom[I]{제\textbf{\color{magenta}가}}{je\textbf{\color{magenta}ga}}
            \rom[God's]{신\textbf{\color{orange}의}}{sin\textbf{\color{orange}ui}}
            \rom[apple]{사과\textbf{\color{blue}를}}{sagwa\textbf{\color{blue}reul}} 
            \rom[eat]{먹어요}{meogeoyo} &
            \textbf{\color{magenta}I} eat God\textbf{\color{orange}'s} \textbf{\color{blue}apple}
            \\ \hline
        Location &
            에서 &
            \rom[I]{제\textbf{\color{magenta}가}}{je\textbf{\color{magenta}ga}}
            \rom[at the garden]{정원\textbf{\color{purple}에서}}{jeongwon\textbf{\color{purple}eseo}}
            \rom[apple]{사과\textbf{\color{blue}를}}{sagwa\textbf{\color{blue}reul}}
            \rom[eat]{먹어요}{meogeoyo} &
            \textbf{\color{magenta}I} eat an \textbf{\color{blue}apple} \textbf{\color{purple}at} the garden

    \end{tabular}
\end{center}

In informal speech, these particles are often dropped if the context is clear. Also note that there are much more particles than the ones in this table, however, they will be explained in their respective sections.
