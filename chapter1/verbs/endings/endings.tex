\subsection{Endings}

When conjugating a verb, we always take into account the dictionary form of a word, which will always end in \rom[]{다}{da}, this is what we call in English the \textit{citation form} or \textit{lemma}, maybe you could compare it with the "to" in words like "to go", "to eat", and so on. If we remove this 다, we get what we call the \textit{stem} of the verb.

\subsubsection{Grammatical Mood Endings}
The English language is quite reserved with these endings, you have present, past, past participle and continous, and in order to express these moods, we use an auxiliary verb at the start of the verb (I \textit{should} go, I \textit{was} going, I \textit{could} go, and so on). In Korean, there are different endings for each grammatical mood, and it could be quite tedious compared to English.

Depending on the grammatical mood \rom[form]{형}{hyeong} of the sentence, each level has a specific ending. The four most important grammatical moods and the ones explained in further detail are \textit{indicative, interrogative, imperative and propositive}.

In order to make this text as light as possible, the grammatical rules won't be explained with much detail\footnote{I would like to justify my lazyness by saying that there are over 40 basic endings and over 400 different combinations}, it is recommended to use sites or conjugating dictionaries that are built for that purpose, and trying to build knowledge by experience rather by memorizing some \textit{very} specific rules.

Here are some tables with these grammatical moods in action with its most simple rules:

\begin{tcolorbox}[box=Declarative 평서형]
\begin{tabular}{c|>{\centering\arraybackslash}m{2.5cm}|>{\centering\arraybackslash}m{\dimexpr0.3\linewidth}|>{\arraybackslash}m{5cm}}
    \textbf{Level} & \textbf{Ending} & \textbf{Rule} (ends with) & \textbf{Examples} (present)
    \\\hline
    \textbf{IL} 
        & \shortstack{--아 \\ --어} 
        & \shortstack{bright vowel $\rightarrow$ --아 \\ 
                      otherwise $\rightarrow$ --어} 
        & \spacedstack{
            \rom[]{\textbf{\color{magenta}가}\color{blue}다}{\textbf{\color{magenta}ga}\color{blue}da}$\rightarrow$\rom[]{가}{ga}\\
            \rom[]{\textbf{\color{violet}먹}\color{blue}다}{\textbf{\color{violet}meog}\color{blue}da}$\rightarrow$\rom[]{먹어}{meogeo}
        }
    \\\hline
    \textbf{IH}
        & \shortstack{--아요 \\ --어요 }
        & \shortstack{bright vowel $\rightarrow$ --아요 \\ 
                      otherwise $\rightarrow$ --어요} 
        & \spacedstack{
            \rom[]{\textbf{\color{magenta}가}\color{blue}다}{\textbf{\color{magenta}ga}\color{blue}da}$\rightarrow$\rom[]{가요}{gayo}\\
            \rom[]{\textbf{\color{violet}먹}\color{blue}다}{\textbf{\color{violet}meog}\color{blue}da}$\rightarrow$\rom[]{먹어요}{meogeoyo}
        }
    \\\hline
    \textbf{FL}
        & \shortstack{--ㄴ아 \\ --응다 }
        & \shortstack{vowel $\rightarrow$ --ㄴ다 \\ 
                      consonant $\rightarrow$ --는다} 
        & \spacedstack{
            \rom[]{\textbf{\color{magenta}가}\color{blue}다}{\textbf{\color{magenta}ga}\color{blue}da}$\rightarrow$\rom[]{간다}{ganda}\\
            \rom[]{\textbf{\color{violet}먹}\color{blue}다}{\textbf{\color{violet}meog}\color{blue}da}$\rightarrow$\rom[]{먹는다}{meogneunda}
        }
    \\\hline
    \textbf{FH}
        & \shortstack{--ㅂ니다 \\ --습니다 }
        & \shortstack{vowel $\rightarrow$ --ㅂ니다 \\ 
                      consonant $\rightarrow$ --습니다} 
        & \spacedstack{
            \rom[]{\textbf{\color{magenta}가}\color{blue}다}{\textbf{\color{magenta}ga}\color{blue}da}$\rightarrow$\rom[]{갑니다}{gabnida}\\
            \rom[]{\textbf{\color{violet}먹}\color{blue}다}{\textbf{\color{violet}meog}\color{blue}da}$\rightarrow$\rom[]{먹습니다}{meogseubnida}
        }
\end{tabular}
\end{tcolorbox}
\begin{tcolorbox}[box=Interrogative 의문형]
\begin{tabular}{c|>{\centering\arraybackslash}m{2.5cm}|>{\centering\arraybackslash}m{\dimexpr0.3\linewidth}|>{\arraybackslash}m{5cm}}
    \textbf{Level} & \textbf{Ending} & \textbf{Rule} (ends with) & \textbf{Examples} (present)
    \\\hline
    \textbf{IL} 
        & \shortstack{--니? \\ --어?}
        & \shortstack{bright vowel $\rightarrow$ --니? \\ 
                      otherwise $\rightarrow$ --어?} 
        & \spacedstack{
            \rom[]{\textbf{\color{magenta}가}\color{blue}다}{\textbf{\color{magenta}ga}\color{blue}da}$\rightarrow$\rom[]{가니?}{gani?}\\
            \rom[]{\textbf{\color{violet}먹}\color{blue}다}{\textbf{\color{violet}meog}\color{blue}da}$\rightarrow$\rom[]{먹어?}{meogeo?}
        }
    \\\hline
    \textbf{IH}
        & \shortstack{--아요? \\ --어요? }
        & \shortstack{bright vowel $\rightarrow$ --아요? \\ 
                      otherwise $\rightarrow$ --어요?} 
        & \spacedstack{
            \rom[]{\textbf{\color{magenta}가}\color{blue}다}{\textbf{\color{magenta}ga}\color{blue}da}$\rightarrow$\rom[]{가요?}{gayo?}\\
            \rom[]{\textbf{\color{violet}먹}\color{blue}다}{\textbf{\color{violet}meog}\color{blue}da}$\rightarrow$\rom[]{먹어요?}{meogeoyo?}
        }
    \\\hline
    \textbf{FL}
        & \shortstack{--느냐? \\ --냐?}
        & \shortstack{vowel $\rightarrow$ --느냐? \\ 
                      consonant $\rightarrow$ --냐?} 
        & \spacedstack{
            \rom[]{\textbf{\color{magenta}가}\color{blue}다}{\textbf{\color{magenta}ga}\color{blue}da}$\rightarrow$\rom[]{가냐?}{ganya?}\\
            \rom[]{\textbf{\color{violet}먹}\color{blue}다}{\textbf{\color{violet}meog}\color{blue}da}$\rightarrow$\rom[]{먹느냐?}{meogneunya}
        }
    \\\hline
    \textbf{FH}
        & \shortstack{--ㅂ니까? \\ --습니까? }
        & \shortstack{vowel $\rightarrow$ --ㅂ니까? \\ 
                      consonant $\rightarrow$ --습니까?} 
        & \spacedstack{
            \rom[]{\textbf{\color{magenta}가}\color{blue}다}{\textbf{\color{magenta}ga}\color{blue}da}$\rightarrow$\rom[]{갑니까?}{gabnikka}\\
            \rom[]{\textbf{\color{violet}먹}\color{blue}다}{\textbf{\color{violet}meog}\color{blue}da}$\rightarrow$\rom[]{먹습니까?}{meogseubnikka?}
        }
\end{tabular}
\end{tcolorbox}
\begin{tcolorbox}[box=Imperative 명령형]
\begin{tabular}{c|>{\centering\arraybackslash}m{2.5cm}|>{\centering\arraybackslash}m{\dimexpr0.3\linewidth}|>{\arraybackslash}m{5cm}}
    \textbf{Level} & \textbf{Ending} & \textbf{Rule} (ends with) & \textbf{Examples} (present)
    \\\hline
    \textbf{IL} 
        & \shortstack{--아라 \\ --어라}
        & \shortstack{bright vowel $\rightarrow$ --아라 \\ 
                      otherwise $\rightarrow$ --어라} 
        & \spacedstack{
            \rom[]{\textbf{\color{magenta}가}\color{blue}다}{\textbf{\color{magenta}ga}\color{blue}da}$\rightarrow$\rom[]{가라}{gala}\\
            \rom[]{\textbf{\color{violet}먹}\color{blue}다}{\textbf{\color{violet}meog}\color{blue}da}$\rightarrow$\rom[]{먹어라}{meogeola}
        }
    \\\hline
    \textbf{IH}
        & \shortstack{--세요 \\ --으세요 }
        & \shortstack{bright vowel $\rightarrow$ --세요 \\ 
                      otherwise $\rightarrow$ --으세요} 
        & \spacedstack{
            \rom[]{\textbf{\color{magenta}가}\color{blue}다}{\textbf{\color{magenta}ga}\color{blue}da}$\rightarrow$\rom[]{가세요}{gaseyo}\\
            \rom[]{\textbf{\color{violet}먹}\color{blue}다}{\textbf{\color{violet}meog}\color{blue}da}$\rightarrow$\rom[]{먹으세요}{meogeuseyo}
        }
    \\\hline
    \textbf{FL}\footnote{Note that, in this grammatical mood, it sounds exactly the same as in \textit{informal low}}
        & \shortstack{--아라 \\ --어라}
        & \shortstack{vowel $\rightarrow$ --아라 \\ 
                      consonant $\rightarrow$ --어라} 
        & \spacedstack{
            \rom[]{\textbf{\color{magenta}가}\color{blue}다}{\textbf{\color{magenta}ga}\color{blue}da}$\rightarrow$\rom[]{가라}{gala}\\
            \rom[]{\textbf{\color{violet}먹}\color{blue}다}{\textbf{\color{violet}meog}\color{blue}da}$\rightarrow$\rom[]{먹어라}{meogeola}
        }
    \\\hline
    \textbf{FH}
        & \shortstack{--십시오 \\ --으십시오 }
        & \shortstack{vowel $\rightarrow$ --십시오 \\ 
                      consonant $\rightarrow$ --으십시오} 
        & \spacedstack{
            \rom[]{\textbf{\color{magenta}가}\color{blue}다}{\textbf{\color{magenta}ga}\color{blue}da}$\rightarrow$\rom[]{가십시오}{gasibsio}\\
            \rom[]{\textbf{\color{violet}먹}\color{blue}다}{\textbf{\color{violet}meog}\color{blue}da}$\rightarrow$\rom[]{먹으십시오}{meogeosibsio}
        }
\end{tabular}
\end{tcolorbox}
\begin{tcolorbox}[box=Propositive 청유형]
\begin{tabular}{c|>{\centering\arraybackslash}m{2.5cm}|>{\centering\arraybackslash}m{\dimexpr0.3\linewidth}|>{\arraybackslash}m{5cm}}
    \textbf{Level} & \textbf{Ending} & \textbf{Rule} (ends with) & \textbf{Examples} (present)
    \\\hline
    \textbf{IL} 
        & \shortstack{--자}
        & \shortstack{} 
        & \spacedstack{
            \rom[]{\textbf{\color{magenta}가}\color{blue}다}{\textbf{\color{magenta}ga}\color{blue}da}$\rightarrow$\rom[]{가자}{gaja}\\
            \rom[]{\textbf{\color{violet}먹}\color{blue}다}{\textbf{\color{violet}meog}\color{blue}da}$\rightarrow$\rom[]{먹자}{meogja}
        }
    \\\hline
    \textbf{IH}
        & \shortstack{--아요 \\ --어요 }
        & \shortstack{bright vowel $\rightarrow$ --아요 \\ 
                      otherwise $\rightarrow$ --어요} 
        & \spacedstack{
            \rom[]{\textbf{\color{magenta}가}\color{blue}다}{\textbf{\color{magenta}ga}\color{blue}da}$\rightarrow$\rom[]{가요}{gayo}\\
            \rom[]{\textbf{\color{violet}먹}\color{blue}다}{\textbf{\color{violet}meog}\color{blue}da}$\rightarrow$\rom[]{먹어요}{meogeoyo}
        }
    \\\hline
    \textbf{FL}
        & \shortstack{--게}
        & \shortstack{} 
        & \spacedstack{
            \rom[]{\textbf{\color{magenta}가}\color{blue}다}{\textbf{\color{magenta}ga}\color{blue}da}$\rightarrow$\rom[]{가게}{gage}\\
            \rom[]{\textbf{\color{violet}먹}\color{blue}다}{\textbf{\color{violet}meog}\color{blue}da}$\rightarrow$\rom[]{먹게}{meogge}
        }
    \\\hline
    \textbf{FH}
        & \shortstack{--ㅂ시다 \\ --읍시다 }
        & \shortstack{vowel $\rightarrow$ --ㅂ시다 \\ 
                      consonant $\rightarrow$ --읍시다} 
        & \spacedstack{
            \rom[]{\textbf{\color{magenta}가}\color{blue}다}{\textbf{\color{magenta}ga}\color{blue}da}$\rightarrow$\rom[]{갑시다}{gabnida}\\
            \rom[]{\textbf{\color{violet}먹}\color{blue}다}{\textbf{\color{violet}meog}\color{blue}da}$\rightarrow$\rom[]{먹읍시다}{meogeubnida}
        }
\end{tabular}
\end{tcolorbox}
