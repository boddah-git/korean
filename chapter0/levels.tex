\section{Speech Levels}
All verbs conjugations in the Korean language have distinct paradigms depending on the level of formality \textit{(informal vs formal)} and politeness \textit{(low vs high)} towards the listener, which are the following:

\paragraph{Higher Levels}
\begin{description}
    \item[\rom{하소서체}{hasoseo-che}]\footnote{\textbf{Etymology}\rom{체}{che} Comes from the word \textit{style}, \textit{form} or \textit{body} in Sino-Korean.}\footnote{\textbf{Etymology} \rom{하}{ha} Comes from the non-honorific imperative form of the verb \textit{to do} \rom{하다}{hada}} Very formally polite, used to address royalty or in religious texts.
    \item[\rom{하십시오체}{hasipsio-che}] Formally polite, used to address colleagues in formal settings or between strangers at the start of a conversation.
\end{description}

\paragraph{Middle levels}
\begin{description}
    \item [\rom{해요체}{haeyo-che}] Casually polite, used between strangers and colleagues.
    \item [\rom{하오체}{hao-che}] Formally neutral, used in signs or among civil servants and the older generation.
    \item [\rom{하게체}{hage-che}] Neutral, used for those under one's authority.
\end{description}

\paragraph{Lower levels}
\begin{description}
    \item [\rom{해라체}{haera-che}] Formally impolite, used with close friends or relatives and by adults to children.
    \item [\rom{해체}{hae-che}] Casually impolite or intimate, Between close friends and relatives.
\end{description}

However, there are a lot of levels that are either archaic or contextually limited, so this text will be focusing only on: \rom{해체}{hae-che}, \rom{해라체}{haera-che}, \rom{해요체}{haeyo-che}, \rom{하십시오체}{hasipsio-che} and will be called \textit{informal low}, \textit{formal low}, \textit{informal high} and \textit{formal high} respectively for convenience and because they cover 99\% of real-life usage.

Also it is worth warning that this text will try to use the least amount of words possible. It is recommended to use a dictionary, texts or conversations in order to expand one's vocabulary. This text should be used in order to learn how to structure said vocabulary.