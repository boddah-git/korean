\section{Pronouns}
Pronouns are words that replace nouns in a sentence. In Korean, they can be divided into three groups: \textbf{personal}, \textbf{demonstrative} and \textbf{interrogative}.

\begin{itemize}
    \item \textbf{Personal pronouns} are words that are associated to a specific grammatical person (\textit{I, you, they, he, etc.}). 
    
    These words can be categorized in two sections: the person (\textit{first, second or third}) and the number (\textit{singular or plural}), in which the latter can be optionally subdivided into grammatical gender (masculine, femenine or object). It is worth noting that native speakers generally avoid third-person pronouns and are rarely used in spoken conversation.
    
    In Korean, personal pronouns are divided between \textbf{informal} and \textbf{polite or formal} context.

    \begin{center}
        \begin{tcolorbox}[box=Personal Pronouns]
            \begin{center}
                \begin{tabular}{cc|c|c}
                    \textbf{Person} & \textbf{Number} & \textbf{Informal} & \textbf{Formal} \\
                    \hline
                    1st & Singular & \rom[I]{나}{na} & \rom{저}{jeo} \\
                    \hline
                    2nd & Singular & \rom[You]{너}{neo} & \rom[]{그쪽}{geujjok} \\
                    \hline
                    3rd & Singular (General/Masculine) & \rom[He]{그}{geu} & \rom{그분}{geubun} \\
                    \hline
                    & Singular (Femenine)\footnote{This word was created in order to translate third person pronouns from foreign countries, again, third-person pronouns are rarely used in spoken conversation} & \rom[She]{그녀}{geunyeo} & \rom{그분}{geubun} \\
                    \hline
                    & Singular (Object) & \rom[It]{그것}{geugeot} \\
                    \hline
                    1st & Plural & \rom[We]{우리}{uri} & \rom{저희}{jeohui} \\
                    \hline
                    2nd & Plural & \rom[You]{너희}{neohui} & \rom{여러분}{yeoreobun} \\
                    \hline
                    3rd & Plural & \rom[They]{그들}{geudeul} & \rom{그분들}{geubundeul} \\
                \end{tabular}
            \end{center}
        \end{tcolorbox}
    \end{center}
    
    \item \textbf{Demonstrative pronouns} are words that point to specific things\footnote{\textbf{Etymology} \rom[]{것}{geot} comes from the word \textit{thing}, which is used in the third-person singular pronoun for inanimate objects and all of the demonstrative pronouns.} (\textit{this, that, these or those}). 
    
    These words can be categorized in two sections: the distance (\textit{proximal, medial or distal}) and the number (\textit{singular or plural}). 
    
    Note that in English there aren't any differences between medial and distal demosntrative pronouns, although one could accentuate the distance by adding "over there" to the pronoun (\textit{that} versus \textit{that over there}).
    
    \begin{center}
        \begin{minipage}[t]{0.7\textwidth}
        \begin{tcolorbox}[box=Demonstrative Pronouns]
            \begin{center}
                \begin{tabular}{cc|c}
                    \textbf{Distance} & \textbf{Number} & \\
                    \hline
                    Proximal & Singular & \rom[this]{이것}{igeot} \\
                    \hline
                    Medial & Singular & \rom[that]{그것}{geugeot} \\
                    \hline
                    Distal & Singular & \rom[that]{저것}{jeogeot} \\
                    \hline
                    Proximal & PL & \rom[these]{이것들}{igeotdeul}\footnote{\textbf{Note} \rom[]{들}{deul} is a \textit{plural marker}} \\
                    \hline
                    Medial & Plural & \rom[those]{그것들}{geugeotdeul} \\
                    \hline
                    Distal & Plural & \rom[those]{저것들}{jeogeotdeul} \\
        
                \end{tabular}
            \end{center}
        \end{tcolorbox}
        \end{minipage}
    \end{center}
    
    \item \textbf{Interrogative pronouns} are words used to ask a question, (i.e \textit{what, which, when, etc\dots})

    \begin{center}
        \begin{minipage}[t]{0.7\textwidth}
        \begin{tcolorbox}[box=Demonstrative Pronouns]
            \begin{center}
                \begin{tabular}{c|c}
                    \textbf{Function} & \\
                    \hline
                    Person & \rom[who]{누구}{nugu} \\
                    \hline
                    Thing & \rom[what]{무엇}{mueos}/\rom[what]{뭐}{mua}\footnote{무엇 is more formal} \\
                    \hline
                    Place & \rom[where]{어디}{eodi} \\
                    \hline
                    Time & \rom[when]{언제}{eonje} \\
                \end{tabular}
                \hfill
                \begin{tabular}{c|c}
                    \textbf{Function} & \\
                    \hline
                    Reason & \rom[why]{왜}{wae} \\
                    \hline
                    Manner & \rom[how]{어떻게}{eotteohge} \\
                    \hline
                    Selection & \rom[which]{어느}{eoneu} \\
                    \hline
                    Quantity & \rom[how many]{몇}{myeoch} \\
        
                \end{tabular}
            \end{center}
        \end{tcolorbox}
        \end{minipage}
    \end{center}

\end{itemize}




