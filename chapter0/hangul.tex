\section{Korean Alphabet (한글)}
\subsection{Introduction}
Hangul (한글) is the writing system used for the Korean language. Similarly to Latin, it is a \textbf{phonetic alphabet} unlike Chinese or Japanese, which are logographic scripts. 

\subsection{Basic Structure}
Each character represents a \textbf{syllable} or \textbf{syllabic block}, and each syllable is composed of individual letters called \textbf{jamo (자므)}, which represent consonants or \textbf{jaeum (자음)} and vowels or \textbf{moeum (모음)}, read from left to right and top to bottom.

There are a total of 14 basic consonants and 10 monophthong vowels, following 5 double consonants and 11 additional vowels formed by combining two monophthongs in the same syllable (diphthongs).

Each syllable typically consists of:

\begin{itemize}
    \item An initial consonant \textbf{(Choseong 초성)}
    \item A medial vowel \textbf{(Jungseong 중성)}
    \item An optional final consonant \textbf{(Jongseong 종성)}
\end{itemize}

Something to note is that the vowels ㅏ,ㅓ,ㅑ,ㅕ,ㅐ,ㅔ,ㅖ and ㅒ are written to the \textbf{right} of the initial consonant, while the rest are written to the \textbf{bottom}.

Throughout this text, in order to facilitate the learning process, all Hangul writing will be written with the following format:

\begin{center}
    \rom{안녕하세요}{annyeonghaseyo}
\end{center}

Or, in case there is the need for translation:

\begin{center}
    \rom[hello]{안녕하세요}{annyeonghaseyo}
\end{center}

\newpage
Here's a table for the possible initial, medial and final letters: 

\begin{minipage}[t]{0.45\textwidth}
    \vspace{0.5em}
    \begin{tcolorbox}[box=Initials 초성]
        \begin{tabularx}
            {\textwidth}
            {   >{\centering\arraybackslash}X
                >{\centering\arraybackslash}X
                >{\centering\arraybackslash}X
                >{\centering\arraybackslash}X
            }
            \textbf{한글} & \textbf{Latin} & \textbf{IPA} & \textbf{Key}    \\
            \hline
            ㄱ  & g              & \ipa{k}      & \keys{D}        \\
            ㄲ  & kk             & \ipa{k͈}      & \keys{\shift D} \\
            ㄴ  & n              & \ipa{n}      & \keys{S}        \\
            ㄷ  & d              & \ipa{t}      & \keys{E}        \\
            ㄸ  & tt             & \ipa{t͈}      & \keys{\shift E} \\
            ㄹ  & r              & \ipa{ɾ}      & \keys{F}        \\
            ㅁ  & m              & \ipa{m}      & \keys{A}        \\
            ㅂ  & b              & \ipa{p}      & \keys{Q}        \\
            ㅃ  & pp             & \ipa{p͈}      & \keys{\shift Q} \\
            ㅅ  & s              & \ipa{s}      & \keys{T}        \\
            ㅆ  & ss             & \ipa{s͈}      & \keys{\shift T} \\
            ㅇ  &                &              & \keys{X}        \\
            ㅈ  & j              & \ipa{tɕ}     & \keys{W}        \\
            ㅉ  & jj             & \ipa{tɕ͈}     & \keys{\shift W} \\
            ㅊ  & ch             & \ipa{tɕʰ}    & \keys{C}        \\
            ㅋ  & k              & \ipa{kʰ}     & \keys{Z}        \\
            ㅌ  & t              & \ipa{tʰ}     & \keys{X}        \\
            ㅍ  & p              & \ipa{pʰ}     & \keys{V}        \\
            ㅎ  & h              & \ipa{h}      & \keys{G}        \\
        \end{tabularx}
    \end{tcolorbox}
\end{minipage}%
\hfill
\begin{minipage}[t]{0.45\textwidth}
    \vspace{0.5em}
    \begin{tcolorbox}[box=Medials 중성]
        \begin{tabularx}
            {\textwidth}
            {   >{\centering\arraybackslash}X
                >{\centering\arraybackslash}X
                >{\centering\arraybackslash}X
                >{\centering\arraybackslash}X
            }
            \textbf{한글} & \textbf{Latin} & \textbf{IPA} & \textbf{Key} \\
            \hline
            ㅏ  & a              & \ipa{a}      & \keys{K}          \\
            ㅐ  & ae             & \ipa{ɛ}      & \keys{O}          \\
            ㅑ  & ya             & \ipa{ja}     & \keys{I}          \\
            ㅒ  & yae            & \ipa{jɛ}     & \keys{\shift O}   \\
            ㅓ  & eo             & \ipa{ʌ}      & \keys{J}          \\
            ㅔ  & e              & \ipa{e}      & \keys{P}          \\
            ㅕ  & yeo            & \ipa{jʌ}     & \keys{U}          \\
            ㅖ  & ye             & \ipa{je}     & \keys{\shift P}   \\
            ㅗ  & o              & \ipa{o}      & \keys{H}          \\
            ㅘ  & wa             & \ipa{wa}     &                   \\
            ㅙ  & wae            & \ipa{wɛ}     &                   \\
            ㅚ  & oe             & \ipa{ø}      &                   \\
            ㅛ  & yo             & \ipa{jo}     & \keys{Y}          \\
            ㅜ  & u              & \ipa{u}      & \keys{N}          \\
            ㅝ  & wo             & \ipa{wo}     &                   \\
            ㅞ  & we             & \ipa{wɛ}     &                   \\
            ㅟ  & wi             & \ipa{wi}     &                   \\
            ㅠ  & yu             & \ipa{ju}     & \keys{B}          \\
            ㅡ  & eu             & \ipa{ɯ}      & \keys{M}          \\
            ㅢ  & ui             & \ipa{ɯi}     &                   \\
            ㅣ  & i              & \ipa{i}      & \keys{L}          \\
        \end{tabularx}
    \end{tcolorbox}
\end{minipage}
\hfill

\begin{tcolorbox}[box=Finals 종성 \textbf{(Optional)}]
    \noindent
    \begin{minipage}[t]{0.48\textwidth}
        \centering
        \textbf{Normal Finals}
        \vspace{0.5em}

        \begin{tabularx}{0.95\textwidth}{
                >{\centering\arraybackslash}X
                >{\centering\arraybackslash}X
                >{\centering\arraybackslash}X}
            \textbf{한글} & \textbf{Latin} & \textbf{IPA} \\
            \hline
            ㄱ  & g              & \ipa{k̚}      \\
            ㄴ  & n              & \ipa{n}      \\
            ㄷ  & d              & \ipa{t̚}      \\
            ㄹ  & l              & \ipa{l}      \\
            ㅁ  & m              & \ipa{m}      \\
            ㅂ  & b              & \ipa{p̚}      \\
            ㅅ  & s              & \ipa{t̚}      \\
            ㅇ  & ng             & \ipa{ŋ}      \\
            ㅈ  & j              & \ipa{t̚}      \\
            ㅊ  & ch             & \ipa{t̚}      \\
            ㅋ  & k              & \ipa{k̚}      \\
            ㅌ  & t              & \ipa{t̚}      \\
            ㅍ  & p              & \ipa{p̚}      \\
            ㅎ  & h              & \ipa{t̚}      \\
        \end{tabularx}
    \end{minipage}
    \hfill
    \begin{minipage}[t]{0.48\textwidth}
        \centering
        \textbf{Double/Complex Finals}
        \vspace{0.5em}

        \begin{tabularx}{0.95\textwidth}{
                >{\centering\arraybackslash}X
                >{\centering\arraybackslash}X
                >{\centering\arraybackslash}X}
            \textbf{한글} & \textbf{Latin} & \textbf{IPA} \\
            \hline
            ㄲ  & kk             & \ipa{k̚}      \\
            ㄳ  & gs             & \ipa{k̚}      \\
            ㄵ  & nj             & \ipa{n}      \\
            ㄶ  & nh             & \ipa{n}      \\
            ㄺ  & lg             & \ipa{k̚}      \\
            ㄻ  & lm             & \ipa{m}      \\
            ㄼ  & lb             & \ipa{p̚}      \\
            ㄽ  & ls             & \ipa{t̚}      \\
            ㄾ  & lt             & \ipa{t̚}      \\
            ㄿ  & lp             & \ipa{p̚}      \\
            ㅄ  & bs             & \ipa{p̚}      \\
            ㅆ  & ss             & \ipa{t̚}      \\
        \end{tabularx}
    \end{minipage}
\end{tcolorbox}

\subsection{Vowels}
\subsubsection{Bright, Dark and Neutral Vowels}
Hangul vowels follow a certain harmony which is applied when forming them

\begin{description}
    \item[Bright Vowels \rom{양성 모음}{yangseong moeum}]: ㅏ, ㅗ and ㆍ\footnote{ㆍ is an extinct character called \rom[lower a]{아래아}{araea}}
    \item[Dark Vowels \rom{음성 모음}{eumseong moeum}]: ㅓ,ㅜ and ㅡ\footnote{ㅡ is considered both partially dark and partially neutral}
    \item[Neutral Vowels \rom{중성 모음}{jungseong moeum}]: ㅣ 
\end{description}

This is really useful for when trying to study etymologies or sound symbolism, since words were often associated what type of vowel do they use\footnote{Fun fact: \rom[man]{남자}{namja} uses bright vowels and \rom[woman]{여자}{yeoja} uses dark vowels}.

Here we can see how compound monophthongs are formed:

\begin{multicols}{3}
    \begin{enumerate}
        \item [ㅐ] = ㅏ + ㅣ (Bright)
        \item [ㅔ] = ㅓ + ㅣ (Dark)
        \item [ㅒ] = ㅣ + ㅒ (Bright)
        \item [ㅖ] = | + ㅔ (Dark)
    \end{enumerate}
\end{multicols}

Then, based on these "basic vowels", we have the rest of the diphthongs as a combination of bright+bright or dark+dark:
\begin{multicols}{3}{
    \begin{enumerate}
    \item [ㅛ] = | + ㅗ (Bright)
    \item [ㅕ] = ㅣ + ㅓ (Dark)
    \item [ㅑ] = ㅣ + ㅏ (Bright)
    \item [ㅠ] = ㅣ + ㅜ (Dark)
    \item [ㅘ] = ㅗ + ㅏ (Bright)
    \item [ㅙ] = ㅗ + ㅐ (Bright)
    \item [ㅚ] = ㅗ + ㅣ (Bright)
    \item [ㅝ] = ㅜ + ㅓ (Dark)
    \item [ㅞ] = ㅜ + ㅔ (Dark)
    \item [ㅟ] = ㅜ + ㅣ (Dark)
    \item [ㅢ] = ㅡ + ㅣ (Dark)
    \end{enumerate}}
\end{multicols}

There are also certain vowels that are considered extinct or obsolete, but you may see them in older texts

\begin{multicols}{3}
    \begin{enumerate}
        \item [ㆎ] = ㆍ + ㅣ (Bright)
        \item [ㆉ] = ㅛ + ㅣ (Bright)
        \item [ㆌ] = ㅠ + ㅣ (Dark)
    \end{enumerate}
\end{multicols}

This text will use this concept when talking about grammatical rules, although it is mostly an interesting fact about how Korean phonology is formed, it is absolutely optional to learn about this and I don't think Korean schools even teach about this concept.