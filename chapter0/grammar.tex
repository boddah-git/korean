\section{Grammar}

Now that we know how to read Korean text, we will need to learn how to process a Korean phrase, which is why we need to understand some linguistical definitions beforehand.

\subsection{Morphenes}

Words are made up of \textit{morphenes} or \textit{morphological units}. A morpheme can be an individual word (free morphenes) or parts of a word (bound morphemenes). Korean is an \textit{agglutinative} language, which means that it combines morphemes in order to form a word.

In Korean, there are three types of morphenes: \textbf{stems} (\rom[]{어근}{eogeun}); \textbf{particles} (\rom[]{조사}{josa}); and \textbf{affixes}\footnote{Affixes are bound morphenes that modify the meaning or grammatical function of a word and are always attached to another morpheme, like "ing" in learn\textit{ing}} (both derivational (\rom[]{접사}{jeobsa}) and inflectional (\rom[]{어미}{eomi})).

\subsection{Function and Lexical Words}

Each word can be broadly classified into two categories: Lexical and Function words:
\begin{itemize}
    \item Lexical are "content" words, which means that they contribute to the meaning of the sentence (nouns, adjectives, numbers, etc.). 
    
    \item Functional words are words that are used to express grammatical relationships among other words within a sentence (pronouns, prepositions, particles, etc.).
\end{itemize}

\subsection{Syntactic Order}

Languages set a specific order when structuring a sentence. Unlike English, which follows an $Subject-Verb-Object$ order, Korean follows $Subject-Object-Verb$ ordering.