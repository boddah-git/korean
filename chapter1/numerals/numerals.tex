\section{Numbering}

Korean has two distinct numbering systems: \textbf{Native and Sino\footnote{Sino refers to Korean words of Chinese origin}}. Each is used in different contexts:

\begin{tcolorbox}[title=Sino vs Native]
    \begin{tabularx}{\textwidth}{
        X
        >{\centering\arraybackslash}X
        >{\centering\arraybackslash}X}
        
         & \textbf{Sino} & \textbf{Native} \\ \hline
        \textbf{Range} & 0--$\infty$ & 1--99 \\
        \textbf{Formality} & More formal & More casual \\
        \textbf{Use} & Abstract, precise things & Familiar, natural things  \\
        \hline\\
        \textbf{Counting} & \cross & \checkmark \\
        \textbf{Hour} & \cross & \checkmark \\
        \textbf{Minutes / Seconds} & \checkmark & \cross \\
        \textbf{Dates} & \checkmark & \cross \\
        \textbf{Phone numbers} & \checkmark & \cross \\
        \textbf{Money} & \checkmark & \cross \\
        \textbf{Floor numbers} & \checkmark & \cross \\
        \textbf{Math / Units} & \checkmark & \cross
    \end{tabularx}
\end{tcolorbox}

\subsection{Sino}
Similarly to English, Sino-Korean combines the numbers 1 to 9 to form greater magnitudes. Here is the list of numbers from 0 to 9:

\begin{multicols}{5}
    \begin{enumerate}
        \setcounter{enumi}{-1}
        \item \rom{영}{yeong}
        \item \rom{일}{il}
        \item \rom{이}{i}
        \item \rom{삼}{sam}
        \item \rom{사}{sa}
        \item \rom{오}{o}
        \item \rom{육}{yuk}
        \item \rom{칠}{chil}
        \item \rom{팔}{pal}
        \item \rom{구}{gu}
    \end{enumerate}
\end{multicols}

If we want to add a digit, it is almost exactly like English.  
For example, if we want to say 94, we say ninety-four \textit{(nine-ten-four)}.  
In Korean, we say \rom[9-10-4]{구십사}{gusipsa}.  
This is the list of some magnitudes:

\begin{multicols}{4}
    \begin{enumerate}
        \setcounter{enumi}{9}
        \item \rom{십}{sip} \setcounter{enumi}{99}
        \item \rom{백}{baek} \setcounter{enumi}{999}
        \item \rom{천}{cheon} \setcounter{enumi}{9999}
        \item \rom{만}{man} \setcounter{enumi}{99999}
        \item \rom{십만}{sipman} \setcounter{enumi}{999999}
        \item \rom{백만}{baekman} \setcounter{enumi}{9999999}
        \item \rom{천만}{cheonman} \setcounter{enumi}{99999999}
        \item \rom{억}{eok} 
    \end{enumerate}
\end{multicols}

Note that unlike English, in which we divide every 3 zeroes (100,000 is a hundred \textbf{thousand} ($100*10^3$)), in Korean we divide every 4 zeroes (100,000 is 삽\textbf{만} ($10*10^4$)).

Also it is worth noting that these numbers should be written like we do in English, using the \textbf{Arabic numeric} symbols. This helps when understanding the difference between Sino and Native: If we would intuitively write them using numbers \textit{(12, 5)}, then we use Sino, if we would write them how they are spelled \textit{(twelve, five)} we use Native.

\subsection{Native}
Fortunately, the Native numbering system only goes from 1 to 99, and then it starts counting exactly like Sino.

\begin{multicols}{5}
    \begin{enumerate}
        \item \rom{하나}{hana}
        \item \rom{둘}{dul}
        \item \rom{셋}{set}
        \item \rom{넷}{net}
        \item \rom{다섯}{daseot}
        \item \rom{여섯}{yeoseot}
        \item \rom{일곱}{ilgop}
        \item \rom{여덟}{yeodeol}
        \item \rom{아홉}{ahop}
    \end{enumerate}
\end{multicols}

However, instead of counting magnitudes like in English, since it has such a limited range, it has specific words for 20, 30, 40, and so on

\begin{multicols}{5}
    \begin{enumerate}
        \setcounter{enumi}{9}
        \item \rom{열}{yeol} \setcounter{enumi}{19}
        \item \rom{스물}{seumul} \setcounter{enumi}{29}
        \item \rom{서른}{seoreun} \setcounter{enumi}{39}
        \item \rom{마흔}{maheun} \setcounter{enumi}{49}
        \item \rom{쉰}{swin} \setcounter{enumi}{59}
        \item \rom{예순}{yesun} \setcounter{enumi}{69}
        \item \rom{일흔}{ilheun} \setcounter{enumi}{79}
        \item \rom{여든}{yeodeun} \setcounter{enumi}{89}
        \item \rom{아흔}{aheun}
    \end{enumerate}
\end{multicols}
