\subsection{Declarative Present Tense (평서형현재시제)}

This grammatical tense is used to \textbf{state facts, describe habits} or \textbf{give information}

\begin{description}
    \item[Informal Low]: --아 (if ends with a bright vowel) / --어 (if anything else)
    \item[Informal High]: --아요 (if ends with a bright vowel) / --어요 (if anything else)
    \item[Formal Low]: --는다 (if ends with a consonant) / --ㄴ다 (if ends with a vowel)
    \item[Formal High] : --습니다  (if ends with a consonant) / --ㅂ니다 (if ends with a vowel) 
\end{description}

Here are some examples:

\begin{center}
    \begin{tabular}{c|c|c|c|c}
        \textbf{Verb Stem} & \textbf{Informal Low} & \textbf{Informal High} & \textbf{Formal Low} & \textbf{Formal High}\\
        \hline
        
        \rom[to go]{\textbf{\color{magenta}가}\color{blue}다}{\textbf{\color{magenta}ga}\color{blue}da}             
            & \rom{가}{ga}
            & \rom{가요}{gayo}
            & \rom{간다}{ganda}
            & \rom{갑니다}{gabnida} \\
        \hline

        \rom[to eat]{\textbf{\color{magenta}먹}\color{blue}다}{\textbf{\color{magenta}meog}\color{blue}da} 
            & \rom{먹어}{meogeo}
            & \rom{먹어요}{meogeoyo}
            & \rom{먹는다}{meogneunda}
            & \rom{먹습니다}{meogseubnida} \\
        \hline

        \rom[to drink]{마\textbf{\color{magenta}시}\color{blue}다}{ma\textbf{\color{magenta}sa}\color{blue}da}
            & \rom{마셔}{masyeo}
            & \rom{마셔요}{masyeoyo}
            & \rom{마신다}{masinda}
            & \rom{마십니다}{masibnida}

    \end{tabular}
\end{center}

Korean has several groups of verbs that don't follow these standard conjugation rules, and there are certain verbs that should count as irregular but are not for some reason in particular, it is advised to take these groups with a grain of salt.

Here's a quick cheat sheet for all the groups:

\begin{tabularx}{\linewidth}{X|X}
    \textbf{Vowel after ㄷ--ending stem}
        \begin{itemize}
            \item [Rule]: ㄷ $\rightarrow$ ㄹ
            \item [E.g]: \rom[to walk]{걷다}{geodda}
        \end{itemize}
        \begin{tabular}{c|c|c|c}
            \textbf{IL} & \textbf{IH} & \textbf{FL} & \textbf{FH} \\
            \rom{걸어}{geoleo} & \rom{걸어요}{geoleoyo} & \rom{걷는다}{geodneunda} & \rom{걷습니다}{geodseubnida} 
        \end{tabular}
    & \textbf{Vowel after ㅂ--ending stem}
        \begin{itemize}
            \item [Rule]: ㅂ $\rightarrow$ 우/오\footnote{Left if the verb has dark vowels, right if bright.} (Only in IL/IH)\footnote{It was previously used in Formal High as well, so some dictionaries may accept both as valid}
            \item [E.g]: \rom[to help]{돕다}{dobda}
        \end{itemize}
        \begin{tabular}{c|c|c|c}
            \textbf{IL} & \textbf{IH} & \textbf{FL} & \textbf{FH} \\
            \rom{도와}{dowa} & \rom{도와요}{dowayo} & \rom{돕는다}{dobneunda} & \rom{돕습니다}{dobseubnida} 
        \end{tabular}
    \\ \hline
    \textbf{아 / 어요 after 르 stem}
        \begin{itemize}
            \item [Rule]: Add ㄹ to last syllable
            \item [E.g]: \rom[to differ]{다르다}{daleuda}
        \end{itemize}
        \begin{tabular}{c|c|c|c}
            \textbf{IL} & \textbf{IH} & \textbf{FL} & \textbf{FH} \\
            \rom{달라}{dalla} & \rom{달라요}{dallayo} & \rom{다른다}{daleunda} & \rom{다릅니다}{daleubnida} 
        \end{tabular}
    & \textbf{Vowel after ㅅ--ending stem}
        \begin{itemize}
            \item [Rule]: Drop ㅅ
            \item [E.g]: \rom[to surpass]{낫다}{nasda}
        \end{itemize}
        \begin{tabular}{c|c|c|c}
            \textbf{IL} & \textbf{IH} & \textbf{FL} & \textbf{FH} \\
            \rom{나아}{naa} & \rom{나아요}{naayo} & \rom{낫는다}{nasneunda} & \rom{낫습니다}{nasseubnida}
        \end{tabular}
    \\ \hline
    \textbf{Stems ending in ㅎ\footnote{Usually adjectives or \textit{descriptive verbs}}}
        \begin{itemize}
            \item [Rule]: Drop ㅎ, Use ㅐ (Only in IL/IH)
            \item [E.g]: \rom[to be white]{하얗다}{hayahda}
        \end{itemize}
        \begin{tabular}{c|c|c|c}
            \textbf{IL} & \textbf{IH} & \textbf{FL} & \textbf{FH} \\
            \rom{하얘}{hayae} & \rom{하얘요}{hayaeyo} & \rom{하얗다}{hayahneunda} & \rom{하얗습니다}{hayahseubnida} 
        \end{tabular}
    & \textbf{Stems ending in ㅡ}
        \begin{itemize}
            \item [Rule]: Drop ㅡ, vowel merge (Only in IL/IH)
            \item [E.g]: \rom[to write]{쓰다}{sseuda}
        \end{itemize}
        \begin{tabular}{c|c|c|c}
            \textbf{IL} & \textbf{IH} & \textbf{FL} & \textbf{FH} \\
            \rom{써}{sseo} & \rom{써요}{sseoyo} & \rom{쓴다}{sseunda} & \rom{씁니다}{sseubnida} 
        \end{tabular}
\end{tabularx}
